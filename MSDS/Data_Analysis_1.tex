% Options for packages loaded elsewhere
\PassOptionsToPackage{unicode}{hyperref}
\PassOptionsToPackage{hyphens}{url}
%
\documentclass[
]{article}
\usepackage{lmodern}
\usepackage{amssymb,amsmath}
\usepackage{ifxetex,ifluatex}
\ifnum 0\ifxetex 1\fi\ifluatex 1\fi=0 % if pdftex
  \usepackage[T1]{fontenc}
  \usepackage[utf8]{inputenc}
  \usepackage{textcomp} % provide euro and other symbols
\else % if luatex or xetex
  \usepackage{unicode-math}
  \defaultfontfeatures{Scale=MatchLowercase}
  \defaultfontfeatures[\rmfamily]{Ligatures=TeX,Scale=1}
\fi
% Use upquote if available, for straight quotes in verbatim environments
\IfFileExists{upquote.sty}{\usepackage{upquote}}{}
\IfFileExists{microtype.sty}{% use microtype if available
  \usepackage[]{microtype}
  \UseMicrotypeSet[protrusion]{basicmath} % disable protrusion for tt fonts
}{}
\makeatletter
\@ifundefined{KOMAClassName}{% if non-KOMA class
  \IfFileExists{parskip.sty}{%
    \usepackage{parskip}
  }{% else
    \setlength{\parindent}{0pt}
    \setlength{\parskip}{6pt plus 2pt minus 1pt}}
}{% if KOMA class
  \KOMAoptions{parskip=half}}
\makeatother
\usepackage{xcolor}
\IfFileExists{xurl.sty}{\usepackage{xurl}}{} % add URL line breaks if available
\IfFileExists{bookmark.sty}{\usepackage{bookmark}}{\usepackage{hyperref}}
\hypersetup{
  pdftitle={Data Analysis Assignment \#1 (50 points total)},
  pdfauthor={Prescott, Benjamin},
  hidelinks,
  pdfcreator={LaTeX via pandoc}}
\urlstyle{same} % disable monospaced font for URLs
\usepackage[margin=1in]{geometry}
\usepackage{longtable,booktabs}
% Correct order of tables after \paragraph or \subparagraph
\usepackage{etoolbox}
\makeatletter
\patchcmd\longtable{\par}{\if@noskipsec\mbox{}\fi\par}{}{}
\makeatother
% Allow footnotes in longtable head/foot
\IfFileExists{footnotehyper.sty}{\usepackage{footnotehyper}}{\usepackage{footnote}}
\makesavenoteenv{longtable}
\usepackage{graphicx,grffile}
\makeatletter
\def\maxwidth{\ifdim\Gin@nat@width>\linewidth\linewidth\else\Gin@nat@width\fi}
\def\maxheight{\ifdim\Gin@nat@height>\textheight\textheight\else\Gin@nat@height\fi}
\makeatother
% Scale images if necessary, so that they will not overflow the page
% margins by default, and it is still possible to overwrite the defaults
% using explicit options in \includegraphics[width, height, ...]{}
\setkeys{Gin}{width=\maxwidth,height=\maxheight,keepaspectratio}
% Set default figure placement to htbp
\makeatletter
\def\fps@figure{htbp}
\makeatother
\setlength{\emergencystretch}{3em} % prevent overfull lines
\providecommand{\tightlist}{%
  \setlength{\itemsep}{0pt}\setlength{\parskip}{0pt}}
\setcounter{secnumdepth}{-\maxdimen} % remove section numbering

\title{Data Analysis Assignment \#1 (50 points total)}
\author{Prescott, Benjamin}
\date{}

\begin{document}
\maketitle

\begin{center}\rule{0.5\linewidth}{0.5pt}\end{center}

Submit both the .Rmd and .html files for grading. You may remove the
instructions and example problem above, but do not remove the YAML
metadata block or the first, ``setup'' code chunk. Address the steps
that appear below and answer all the questions. Be sure to address each
question with code and comments as needed. You may use either base R
functions or ggplot2 for the visualizations.

\begin{center}\rule{0.5\linewidth}{0.5pt}\end{center}

The following code chunk will:

\begin{enumerate}
\def\labelenumi{(\alph{enumi})}
\tightlist
\item
  load the ``ggplot2'', ``gridExtra'' and ``knitr'' packages, assuming
  each has been installed on your machine,
\item
  read-in the abalones dataset, defining a new data frame, ``mydata,''
\item
  return the structure of that data frame, and
\item
  calculate new variables, VOLUME and RATIO.
\end{enumerate}

Do not include package installation code in this document. Packages
should be installed via the Console or `Packages' tab. You will also
need to download the abalones.csv from the course site to a known
location on your machine. Unless a \emph{file.path()} is specified, R
will look to directory where this .Rmd is stored when knitting.

\begin{verbatim}
## 'data.frame':    1036 obs. of  8 variables:
##  $ SEX   : chr  "I" "I" "I" "I" ...
##  $ LENGTH: num  5.57 3.67 10.08 4.09 6.93 ...
##  $ DIAM  : num  4.09 2.62 7.35 3.15 4.83 ...
##  $ HEIGHT: num  1.26 0.84 2.205 0.945 1.785 ...
##  $ WHOLE : num  11.5 3.5 79.38 4.69 21.19 ...
##  $ SHUCK : num  4.31 1.19 44 2.25 9.88 ...
##  $ RINGS : int  6 4 6 3 6 6 5 6 5 6 ...
##  $ CLASS : chr  "A1" "A1" "A1" "A1" ...
\end{verbatim}

\begin{center}\rule{0.5\linewidth}{0.5pt}\end{center}

\hypertarget{test-items-starts-from-here---there-are-6-sections}{%
\subsubsection{Test Items starts from here - There are 6
sections}\label{test-items-starts-from-here---there-are-6-sections}}

\hypertarget{section-1-6-points-summarizing-the-data.}{%
\subparagraph{Section 1: (6 points) Summarizing the
data.}\label{section-1-6-points-summarizing-the-data.}}

(1)(a) (1 point) Use \emph{summary()} to obtain and present descriptive
statistics from mydata. Use table() to present a frequency table using
CLASS and RINGS. There should be 115 cells in the table you present.

\begin{verbatim}
##      SEX                LENGTH           DIAM            HEIGHT     
##  Length:1036        Min.   : 2.73   Min.   : 1.995   Min.   :0.525  
##  Class :character   1st Qu.: 9.45   1st Qu.: 7.350   1st Qu.:2.415  
##  Mode  :character   Median :11.45   Median : 8.925   Median :2.940  
##                     Mean   :11.08   Mean   : 8.622   Mean   :2.947  
##                     3rd Qu.:13.02   3rd Qu.:10.185   3rd Qu.:3.570  
##                     Max.   :16.80   Max.   :13.230   Max.   :4.935  
##      WHOLE             SHUCK              RINGS           CLASS          
##  Min.   :  1.625   Min.   :  0.5625   Min.   : 3.000   Length:1036       
##  1st Qu.: 56.484   1st Qu.: 23.3006   1st Qu.: 8.000   Class :character  
##  Median :101.344   Median : 42.5700   Median : 9.000   Mode  :character  
##  Mean   :105.832   Mean   : 45.4396   Mean   : 9.993                     
##  3rd Qu.:150.319   3rd Qu.: 64.2897   3rd Qu.:11.000                     
##  Max.   :315.750   Max.   :157.0800   Max.   :25.000                     
##      VOLUME            RATIO        
##  Min.   :  3.612   Min.   :0.06734  
##  1st Qu.:163.545   1st Qu.:0.12241  
##  Median :307.363   Median :0.13914  
##  Mean   :326.804   Mean   :0.14205  
##  3rd Qu.:463.264   3rd Qu.:0.15911  
##  Max.   :995.673   Max.   :0.31176
\end{verbatim}

\begin{verbatim}
##     
##        3   4   5   6   7   8   9  10  11  12  13  14  15  16  17  18  19  20
##   A1   9   8  24  67   0   0   0   0   0   0   0   0   0   0   0   0   0   0
##   A2   0   0   0   0  91 145   0   0   0   0   0   0   0   0   0   0   0   0
##   A3   0   0   0   0   0   0 182 147   0   0   0   0   0   0   0   0   0   0
##   A4   0   0   0   0   0   0   0   0 125  63   0   0   0   0   0   0   0   0
##   A5   0   0   0   0   0   0   0   0   0   0  48  35  27  15  13   8   8   6
##     
##       21  22  23  24  25
##   A1   0   0   0   0   0
##   A2   0   0   0   0   0
##   A3   0   0   0   0   0
##   A4   0   0   0   0   0
##   A5   4   1   7   2   1
\end{verbatim}

\textbf{Question (1 point): Briefly discuss the variable types and
distributional implications such as potential skewness and outliers.}

\textbf{\emph{Answer: The two variables included are CLASS and RINGS.
RINGS represents the count of rings which is used to determine age (6
rings + 1.5 = years old), where CLASS are groups of abalones based on
their ring counts/age. Based on table it looks to have a positive skew
with no extreme outliers. }}

(1)(b) (1 point) Generate a table of counts using SEX and CLASS. Add
margins to this table (Hint: There should be 15 cells in this table plus
the marginal totals. Apply \emph{table()} first, then pass the table
object to \emph{addmargins()} (Kabacoff Section 7.2 pages 144-147)).
Lastly, present a barplot of these data; ignoring the marginal totals.

\begin{verbatim}
##      
##         A1   A2   A3   A4   A5  Sum
##   F      5   41  121   82   77  326
##   I     91  133   65   21   19  329
##   M     12   62  143   85   79  381
##   Sum  108  236  329  188  175 1036
\end{verbatim}

\includegraphics{Data_Analysis_1_files/figure-latex/Part_1b-1.pdf}

\textbf{Essay Question (2 points): Discuss the sex distribution of
abalones. What stands out about the distribution of abalones by CLASS?}

\textbf{\emph{Answer: The distribution for class A1 and A2 both look to
represent a normal distribution. Classes A3, A4 and A5 all look to
represent a bimodal distribution. It also seems as though the number of
abalones included in the sample vary greatly. Each class shows the
infants collected to be drastically more or less than the male or female
sample counts.}}

(1)(c) (1 point) Select a simple random sample of 200 observations from
``mydata'' and identify this sample as ``work.'' Use
\emph{set.seed(123)} prior to drawing this sample. Do not change the
number 123. Note that \emph{sample()} ``takes a sample of the specified
size from the elements of x.'' We cannot sample directly from
``mydata.'' Instead, we need to sample from the integers, 1 to 1036,
representing the rows of ``mydata.'' Then, select those rows from the
data frame (Kabacoff Section 4.10.5 page 87).

Using ``work'', construct a scatterplot matrix of variables 2-6 with
\emph{plot(work{[}, 2:6{]})} (these are the continuous variables
excluding VOLUME and RATIO). The sample ``work'' will not be used in the
remainder of the assignment.

\includegraphics{Data_Analysis_1_files/figure-latex/Part_1c-1.pdf}

\begin{center}\rule{0.5\linewidth}{0.5pt}\end{center}

\hypertarget{section-2-5-points-summarizing-the-data-using-graphics.}{%
\subparagraph{Section 2: (5 points) Summarizing the data using
graphics.}\label{section-2-5-points-summarizing-the-data-using-graphics.}}

(2)(a) (1 point) Use ``mydata'' to plot WHOLE versus VOLUME. Color code
data points by CLASS.

\includegraphics{Data_Analysis_1_files/figure-latex/Part_2a-1.pdf}

(2)(b) (2 points) Use ``mydata'' to plot SHUCK versus WHOLE with WHOLE
on the horizontal axis. Color code data points by CLASS. As an aid to
interpretation, determine the maximum value of the ratio of SHUCK to
WHOLE. Add to the chart a straight line with zero intercept using this
maximum value as the slope of the line. If you are using the `base R'
\emph{plot()} function, you may use \emph{abline()} to add this line to
the plot. Use \emph{help(abline)} in R to determine the coding for the
slope and intercept arguments in the functions. If you are using ggplot2
for visualizations, \emph{geom\_abline()} should be used.

\includegraphics{Data_Analysis_1_files/figure-latex/Part_2b-1.pdf}

\textbf{Essay Question (2 points): How does the variability in this plot
differ from the plot in (a)? Compare the two displays. Keep in mind that
SHUCK is a part of WHOLE. Consider the location of the different age
classes.}

\textbf{\emph{Answer: The plot in (a) shows that the greater the weight
the more volume. It also looks as though volume is roughly 3 times the
weight. For the plot in (b) the shucked weight is just under half the
whole weight. In plot (a) there is also slightly more variance in the
middle of the weight range in comparison to volume. }}

\begin{center}\rule{0.5\linewidth}{0.5pt}\end{center}

\hypertarget{section-3-8-points-getting-insights-about-the-data-using-graphs.}{%
\subparagraph{Section 3: (8 points) Getting insights about the data
using
graphs.}\label{section-3-8-points-getting-insights-about-the-data-using-graphs.}}

(3)(a) (2 points) Use ``mydata'' to create a multi-figured plot with
histograms, boxplots and Q-Q plots of RATIO differentiated by sex. This
can be done using \emph{par(mfrow = c(3,3))} and base R or
\emph{grid.arrange()} and ggplot2. The first row would show the
histograms, the second row the boxplots and the third row the Q-Q plots.
Be sure these displays are legible.

\includegraphics{Data_Analysis_1_files/figure-latex/Part_3a-1.pdf}

\textbf{Essay Question (2 points): Compare the displays. How do the
distributions compare to normality? Take into account the criteria
discussed in the sync sessions to evaluate non-normality.}

\textbf{\emph{Answer: The male ratio distribution is the closest to
being a normal distribution, but is still positively skewed. The female
distribution has the strongest skew, likely contributed by the number
and strength of the outliers. However, all sexes have outliers
contributing to the non-normality.}}

(3)(b) (2 points) Use the boxplots to identify RATIO outliers (mild and
extreme both) for each sex. Present the abalones with these outlying
RATIO values along with their associated variables in ``mydata'' (Hint:
display the observations by passing a data frame to the kable()
function).

\begin{longtable}[]{@{}llrrrrrrlrr@{}}
\caption{Extreme Outliers}\tabularnewline
\toprule
& SEX & LENGTH & DIAM & HEIGHT & WHOLE & SHUCK & RINGS & CLASS & VOLUME
& RATIO\tabularnewline
\midrule
\endfirsthead
\toprule
& SEX & LENGTH & DIAM & HEIGHT & WHOLE & SHUCK & RINGS & CLASS & VOLUME
& RATIO\tabularnewline
\midrule
\endhead
3 & I & 10.08 & 7.35 & 2.205 & 79.3750 & 44.000 & 6 & A1 & 163.3640 &
0.2693371\tabularnewline
350 & F & 7.98 & 6.72 & 2.415 & 80.9375 & 40.375 & 7 & A2 & 129.5058 &
0.3117620\tabularnewline
\bottomrule
\end{longtable}

\begin{longtable}[]{@{}llrrrrrrlrr@{}}
\caption{Mild Outliers}\tabularnewline
\toprule
& SEX & LENGTH & DIAM & HEIGHT & WHOLE & SHUCK & RINGS & CLASS & VOLUME
& RATIO\tabularnewline
\midrule
\endfirsthead
\toprule
& SEX & LENGTH & DIAM & HEIGHT & WHOLE & SHUCK & RINGS & CLASS & VOLUME
& RATIO\tabularnewline
\midrule
\endhead
3 & I & 10.080 & 7.350 & 2.205 & 79.37500 & 44.00000 & 6 & A1 &
163.364040 & 0.2693371\tabularnewline
37 & I & 4.305 & 3.255 & 0.945 & 6.18750 & 2.93750 & 3 & A1 & 13.242072
& 0.2218308\tabularnewline
42 & I & 2.835 & 2.730 & 0.840 & 3.62500 & 1.56250 & 4 & A1 & 6.501222 &
0.2403394\tabularnewline
58 & I & 6.720 & 4.305 & 1.680 & 22.62500 & 11.00000 & 5 & A1 &
48.601728 & 0.2263294\tabularnewline
67 & I & 5.040 & 3.675 & 0.945 & 9.65625 & 3.93750 & 5 & A1 & 17.503290
& 0.2249577\tabularnewline
89 & I & 3.360 & 2.310 & 0.525 & 2.43750 & 0.93750 & 4 & A1 & 4.074840 &
0.2300704\tabularnewline
105 & I & 6.930 & 4.725 & 1.575 & 23.37500 & 11.81250 & 7 & A2 &
51.572194 & 0.2290478\tabularnewline
200 & I & 9.135 & 6.300 & 2.520 & 74.56250 & 32.37500 & 8 & A2 &
145.027260 & 0.2232339\tabularnewline
350 & F & 7.980 & 6.720 & 2.415 & 80.93750 & 40.37500 & 7 & A2 &
129.505824 & 0.3117620\tabularnewline
420 & F & 11.550 & 7.980 & 3.465 & 150.62500 & 68.55375 & 10 & A3 &
319.365585 & 0.2146560\tabularnewline
458 & F & 11.445 & 8.085 & 3.150 & 139.81250 & 68.49062 & 9 & A3 &
291.478399 & 0.2349767\tabularnewline
746 & M & 13.440 & 10.815 & 1.680 & 130.25000 & 63.73125 & 10 & A3 &
244.194048 & 0.2609861\tabularnewline
754 & M & 10.500 & 7.770 & 3.150 & 132.68750 & 61.13250 & 9 & A3 &
256.992750 & 0.2378764\tabularnewline
803 & M & 10.710 & 8.610 & 3.255 & 160.31250 & 70.41375 & 9 & A3 &
300.153640 & 0.2345924\tabularnewline
810 & M & 12.285 & 9.870 & 3.465 & 176.12500 & 99.00000 & 10 & A3 &
420.141472 & 0.2356349\tabularnewline
852 & M & 11.550 & 8.820 & 3.360 & 167.56250 & 78.27187 & 10 & A3 &
342.286560 & 0.2286735\tabularnewline
870 & M & 11.445 & 8.610 & 2.520 & 99.12500 & 53.70750 & 9 & A3 &
248.324454 & 0.2162795\tabularnewline
\bottomrule
\end{longtable}

\textbf{Essay Question (2 points): What are your observations regarding
the results in (3)(b)?}

\textbf{\emph{Answer: Based on the observations, Infants have the most
mild outliers followed by Males. All mild outliers exist in the first
half of the Age classification range (A1 - A3). However, only an Infant
and Female were extreme outliers existing in the earliest two age
categories (A1, A2). }}

\begin{center}\rule{0.5\linewidth}{0.5pt}\end{center}

\hypertarget{section-4-8-points-getting-insights-about-possible-predictors.}{%
\subparagraph{Section 4: (8 points) Getting insights about possible
predictors.}\label{section-4-8-points-getting-insights-about-possible-predictors.}}

(4)(a) (3 points) With ``mydata,'' display side-by-side boxplots for
VOLUME and WHOLE, each differentiated by CLASS There should be five
boxes for VOLUME and five for WHOLE. Also, display side-by-side
scatterplots: VOLUME and WHOLE versus RINGS. Present these four figures
in one graphic: the boxplots in one row and the scatterplots in a second
row. Base R or ggplot2 may be used.

\includegraphics{Data_Analysis_1_files/figure-latex/Part_4a-1.pdf}

\textbf{Essay Question (5 points) How well do you think these variables
would perform as predictors of age? Explain.}

\textbf{\emph{Answer: I believe the variables would not be a great
predictor of age, as there are too many outliers and overlap between the
age classes. For example, young abalones have some of the highest
recorded volume. Volume would not be a good predictor as the greater or
lesser volume does not clearly distinguish age. The same is true with
whole weight and age, as the oldest recorded class (A5) actually shows a
decrease in weights in comparison to the younger class A4. The younger
classes also have more outliers that are causing skew. }}

\begin{center}\rule{0.5\linewidth}{0.5pt}\end{center}

\hypertarget{section-5-12-points-getting-insights-regarding-different-groups-in-the-data.}{%
\subparagraph{Section 5: (12 points) Getting insights regarding
different groups in the
data.}\label{section-5-12-points-getting-insights-regarding-different-groups-in-the-data.}}

(5)(a) (2 points) Use \emph{aggregate()} with ``mydata'' to compute the
mean values of VOLUME, SHUCK and RATIO for each combination of SEX and
CLASS. Then, using \emph{matrix()}, create matrices of the mean values.
Using the ``dimnames'' argument within \emph{matrix()} or the
\emph{rownames()} and \emph{colnames()} functions on the matrices, label
the rows by SEX and columns by CLASS. Present the three matrices
(Kabacoff Section 5.6.2, p.~110-111). The \emph{kable()} function is
useful for this purpose. You do not need to be concerned with the number
of digits presented.

\begin{longtable}[]{@{}lrrrrr@{}}
\caption{Volume by Class and Sex}\tabularnewline
\toprule
& A1 & A2 & A3 & A4 & A5\tabularnewline
\midrule
\endfirsthead
\toprule
& A1 & A2 & A3 & A4 & A5\tabularnewline
\midrule
\endhead
F & 255.29938 & 276.8573 & 412.6079 & 498.0489 & 486.1525\tabularnewline
I & 66.51618 & 160.3200 & 270.7406 & 316.4129 & 318.6930\tabularnewline
M & 103.72320 & 245.3857 & 358.1181 & 442.6155 & 440.2074\tabularnewline
\bottomrule
\end{longtable}

\begin{longtable}[]{@{}lrrrrr@{}}
\caption{Shucked Weight by Sex and Class}\tabularnewline
\toprule
& A1 & A2 & A3 & A4 & A5\tabularnewline
\midrule
\endfirsthead
\toprule
& A1 & A2 & A3 & A4 & A5\tabularnewline
\midrule
\endhead
F & 38.90000 & 42.50305 & 59.69121 & 69.05161 & 59.17076\tabularnewline
I & 10.11332 & 23.41024 & 37.17969 & 39.85369 & 36.47047\tabularnewline
M & 16.39583 & 38.33855 & 52.96933 & 61.42726 & 55.02762\tabularnewline
\bottomrule
\end{longtable}

\begin{longtable}[]{@{}lrrrrr@{}}
\caption{Ratios by Sex and Class}\tabularnewline
\toprule
& A1 & A2 & A3 & A4 & A5\tabularnewline
\midrule
\endfirsthead
\toprule
& A1 & A2 & A3 & A4 & A5\tabularnewline
\midrule
\endhead
F & 0.1546644 & 0.1554605 & 0.1450304 & 0.1379609 &
0.1233605\tabularnewline
I & 0.1569554 & 0.1475600 & 0.1372256 & 0.1244413 &
0.1167649\tabularnewline
M & 0.1512698 & 0.1564017 & 0.1462123 & 0.1364881 &
0.1262089\tabularnewline
\bottomrule
\end{longtable}

(5)(b) (3 points) Present three graphs. Each graph should include three
lines, one for each sex. The first should show mean RATIO versus CLASS;
the second, mean VOLUME versus CLASS; the third, mean SHUCK versus
CLASS. This may be done with the `base R' \emph{interaction.plot()}
function or with ggplot2 using \emph{grid.arrange()}.

\includegraphics{Data_Analysis_1_files/figure-latex/Part_5b-1.pdf}

\textbf{Essay Question (2 points): What questions do these plots raise?
Consider aging and sex differences.}

\textbf{\emph{Answer: Some questions that these plots raise are: Why do
the weights drop after age class A4? Do infants have their own classes
before becoming an adult? If not, why do they not reach adult weights?
Are the volumes and the shucked weights heavily correlated? }}

5(c) (3 points) Present four boxplots using \emph{par(mfrow = c(2, 2)}
or \emph{grid.arrange()}. The first line should show VOLUME by RINGS for
the infants and, separately, for the adult; factor levels ``M'' and
``F,'' combined. The second line should show WHOLE by RINGS for the
infants and, separately, for the adults. Since the data are sparse
beyond 15 rings, limit the displays to less than 16 rings. One way to
accomplish this is to generate a new data set using subset() to select
RINGS \textless{} 16. Use ylim = c(0, 1100) for VOLUME and ylim = c(0,
400) for WHOLE. If you wish to reorder the displays for presentation
purposes or use ggplot2 go ahead.

\includegraphics{Data_Analysis_1_files/figure-latex/Part_5c-1.pdf}

\textbf{Essay Question (2 points): What do these displays suggest about
abalone growth? Also, compare the infant and adult displays. What
differences stand out?}

\textbf{\emph{Answer: These displays suggest that weight and volume
decrease after about 12 rings. Up to 12 rings the whole weight/volume
increase for both infants and adults. This suggests that at a certain
age the abalones start to lose weight. }}

\begin{center}\rule{0.5\linewidth}{0.5pt}\end{center}

\hypertarget{section-6-11-points-conclusions-from-the-exploratory-data-analysis-eda.}{%
\subparagraph{Section 6: (11 points) Conclusions from the Exploratory
Data Analysis
(EDA).}\label{section-6-11-points-conclusions-from-the-exploratory-data-analysis-eda.}}

\textbf{Conclusions}

\textbf{Essay Question 1) (5 points) Based solely on these data, what
are plausible statistical reasons that explain the failure of the
original study? Consider to what extent physical measurements may be
used for age prediction.}

\textbf{\emph{Answer: One possible reason the study was a failure might
have been how the rings were countered. Based on the background
information, the rings were manually counted which introduces the
possibility of human error or abalones with rings difficult to count.
The sample counts for each sex by class are also uneven, as infant
samples are either significantly more or less than the male and female
samples collected, leading to a biased sample. Both of these make it
difficult in determining patterns that can lead to age prediction based
on sex, class and physical measurements.}}

\textbf{Essay Question 2) (3 points) Do not refer to the abalone data or
study. If you were presented with an overall histogram and summary
statistics from a sample of some population or phenomenon and no other
information, what questions might you ask before accepting them as
representative of the sampled population or phenomenon?}

\textbf{\emph{Answer: If only presented with a histogram and summary
stats, I would ask questions like: ``What is the goal of the data? What
are we trying to determine?'', ``How was the data collected?'', ``What
sampling method was used? Was the sample stratified?'' ``If the sample
was stratified, is the sample an accurate representation of the
strata?'' }}

\textbf{Essay Question 3) (3 points) Do not refer to the abalone data or
study. What do you see as difficulties analyzing data derived from
observational studies? Can causality be determined? What might be
learned from such studies?}

\textbf{\emph{Answer: Some of the issues that arise with observational
studies may be the introduction of a bias due to the researcher not
being able to carry out a randomized experiment, including the general
lack of control with the environment. Because of this, it is also
difficult to determine if the sample is an accurate representation of
the population and how it was collected. Because of these reasons,
confidently determining causality is not able to be determined as a
controlled experiment has not or can not be used. }}

\end{document}
